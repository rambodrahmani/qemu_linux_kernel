Introduciamo un meccanismo di {\em broadcast} tramite il quale
un processo pu\`o inviare un messagio ad un insieme di processi.
Per ricevere o inviare un broadcast i processi si devono preventivamente
registrare come {\em listener} o {\em broadcaster}, rispettivamente.
Un solo processo alla volta pu\`o essere registrato come broadcaster
(un nuovo processo pu\`o diventare broadcaster solo quando il precedente
termina).

Il sistema ricorda tutti i messaggi di broadcast inviati (fino ad un
massimo dato dalla costante \verb|MAX_BROADCAST|) e ciascun processo
listener li riceve tutti, in ordine. I messaggi sono di tipo \verb|natl|.

Per realizzare il sistema aggiungiamo il seguente tipo enumerato:

\begin{verbatim}
   enum broadcast_role { B_NONE, B_BROADCASTER, B_LISTENER };
\end{verbatim}

e il seguente campo ai descrittori di processo:
\begin{verbatim}
  broadcast_role b_reg;
\end{verbatim}
Il campo \`e posto a \verb|B_NONE| alla creazione del processo.

Aggiungiamo infine le seguenti primitive:
\begin{itemize}
   \item \verb|void reg(broadcast_role role)| (tipo 0x3a, da realizzare):
   	registra il processo per il ruolo dato da \verb|role|; \`e un errore se role non specifica n\'e un broadcaster,
	n\'e un listener, se il processo era gi\`a registrato (per lo stesso o un altro ruolo), o se c\`e gi\`a un
	broadcaster e si tenta di registrarne un altro;
   \item \verb|natl listen()| (tipo 0x3b, da realizzare):
   	restituisce il prossimo messagio di broadcast non ancora letto dal processo; se il processo li ha gi\`a
	letti tutti, si blocca in attesa del prossimo; \`e un errore se il processo non \`e registrato come
	listener;
   \item \verb|void broadcast(natl msg)| (tipo 0x3c, da realizzare): 
   	invia in broadcast il messaggio \verb|msg|; \`e un errore se il processo non \`e registrato come broadcaster
	o se il limite di messaggi di broadcast \`e stato superato.
\end{itemize}

Le primitive abortiscono il processo chiamante in caso di errore e tengono conto della priorit\`a tra i processi.

Modificare i file \verb|sistema.cpp| e \verb|sistema.S| in modo da realizzare le primitive mancanti.
Attenzione: il candidato deve definire anche le necessarie strutture dati.
