Introduciamo un meccanismo di {\em broadcast} tramite il quale
un processo pu\`o inviare un messagio ad un insieme di processi.
Per ricevere un broadcast i processi si devono preventivamente
registrare. Un processo pu\`o inviare un brodcast tramite
la primitiva \verb|broadcast(msg)|, la quale attende anche che tutti
i processi che risultano registrati ricevano il messaggio.
Un processo registrato pu\`o ricevere un broadcast invocando
la primitiva \verb|listen()|, che attende che sia disponibile
il prossimo messaggio.

Per realizzare i broadcast introduciamo la seguente struttura dati:
\begin{verbatim}
    struct broadcast {
        int registered;
        int nlisten;
        natl msg;
        proc_elem *listeners;
        proc_elem *broadcaster;
    };
\end{verbatim}
Dove: \verb|registered| \`e il numero di processi registrati;
\verb|nlisten| conta i processi che hanno invocato \verb|listen()| dall'ultimo completamento di una operazione
di broadcast; \verb|msg| contiene l'ultimo messaggio di broadcast; \verb|listeners| \`e la coda
dei processi in attesa del prossimo messaggio; \verb|broadcaster| \`e la coda in cui attende il processo
che vuole inviare il broadcast, in attesa che tutti i processi registrati invochino \verb|listen()|.

Aggiungiamo inoltre le seguenti primitive (abortiscono il processo in caso di errore):
\begin{itemize}
   \item \verb|void reg()| (tipo 0x3a, gi\`a realizzata):
   	registra il processo per la ricezione dei broadcast; non fa niente se il processo \`e gi\`a registrato;
   \item \verb|natl listen()| (tipo 0x3b, da realizzare):
   	riceve il prossimo messagio di broadcast; \`e un errore se il processo non \`e registrato;
   \item \verb|void broadcast(natl msg)| (tipo 0x3c, da realizzare): 
   	invia in broadcast il messaggio \verb|msg|; \`e un errore se il processo \`e registrato.
\end{itemize}

Le primitive abortiscono il processo chiamante in caso di errore e tengono conto della priorit\`a tra i processi.

Per semplicit\`a si assuma che, durante tutta la sua esecuzione, ogni processo provi al pi\`u una sola volta a inviare un broadcast o ad ascoltare un messaggio.
Inoltre, al pi\`u un processo alla volta tenta di eseguire un broadcast.

Modificare i file \verb|sistema.cpp| e \verb|sistema.S| in modo da realizzare le primitive mancanti.
