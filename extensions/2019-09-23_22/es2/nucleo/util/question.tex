Vogliamo fornire ai processi la possibilit\`a di bloccare l'esecuzione di un processo a scelta, ogni volta che questo passa da una certa istruzione.
Per far questo forniamo alcune primitive. Con la primitiva \verb|bpattach(natl id, vaddr rip)| un processo {\em master} 
installa un breakpoint (istruzione \verb|int3|, codice operativo \verb|0xCC|) 
all'indirizzo \verb|rip| nella memoria virtuale del processo \verb|id|, che diventa {\em slave}. Da quel momento in poi il processo slave si blocca se passa da \verb|rip|.
Nel frattempo, usando la primitiva \verb|bpwait()|, il processo master
pu\`o sospendersi in attesa che lo slave passi dal breakpoint.
A quel punto il processo master pu\`o invocare la primitiva \verb|bpcontinue()| per permettere al processo
slave di continuare la propria esecuzione come se non fosse mai stato intercettato. Se lo slave ripassa dal breakpoint
viene intercettato nuovamente e il meccanismo di ripete.
Infine, con la primitiva \verb|bpdetach()|, il processo master rimuove il breakpoint e, se necessario, risveglia un'ultima volta lo slave.

Si noti che  processi che non sono slave non devono essere intercettati. 
Inoltre, se un processo esegue \verb|int3| senza che ci\`o sia stato richiesto da un master, il processo
deve essere abortito.

Aggiungiamo i seguenti campi ai descrittori di processo:
\begin{verbatim}
    des_proc *bp_master;
    des_proc *bp_slave;
    vaddr	bp_addr;
    natb	bp_orig;
    natl	bp_slave_id;
    struct proc_elem  *bp_waiting;
\end{verbatim}
dove: \verb|bp_master| punta al processo master di questo processo (nullo se il
processo non ha un master); \verb|bp_slave| punta al processo slave di
questo processo (nullo se il processo non ha uno slave);
\verb|bp_addr| e \verb|bp_orig| sono sinificativi
solo per i processi slave e contengono, rispettivamente,
l'indirizzo a cui \`e installato il breakpoint e il byte
originariamente contenuto a quell'indirizzo; 
il campo \verb|bp_slave_id| \`e significativo solo per il processo
master e contiene l'id dello slave;
\verb|bp_waiting| \`e una coda su cui i processi master e slave si possono
bloccare: il master su quella dello slave e viceversa.

Si modifichino i file \verb|sistema/sistema.s| e \verb|sistema/sistema.cpp| per implementare le seguenti primitive
(abortiscono il processo in caso di errore):
\begin{itemize}
\item 	\verb|bool bpattach(natl id, vaddr rip)|: (tipo 0x59, gi\`a realizzata): 
	se il processo che invoca la primitiva non \`e uno slave e il processo \verb|id| esiste e non \`e gi\`a uno slave o un master,
	installa il breakpoint all'indirizzo \verb|rip| e restituisce \verb|true|, altrimenti restituisce \verb|false|;
        \`e un errore se \verb|rip| non appartiene all'intervallo $[\mathtt{ini\_utn\_c}, \mathtt{fin\_utn\_c})$ (zona utente/condivisa)
	o se il processo \`e gi\`a master o cerca di diventare master di se stesso.
	\item   \verb|void bpwait()|: (tipo 0x5a, gi\`a realizzata): attende che il processo slave passi dal breakpoint;
	\`e un errore invocare questa primitiva se il processo non \`e master;
	\item \verb|void bpcontinue()|: (tipo 0x5c, da realizzare):  permette allo slave di proseguire l'esecuzione, facendo
	in modo che venga intercettato nuovamente se ripassa dal breakpoint; \`e un errore invocare questa primitiva se il
	processo non \`e master o se lo slave non \`e bloccato sul breakpoint;
\item   \verb|void bpdetach()| (tipo 0x5b, gi\`a realizzata): disfa tutto ci\`o che ha fatto la \verb|bpattach()| e risveglia
	eventualmente il processo slave; \`e un errore invocare questa primitiva se il processo non \`e master;
\end{itemize}

{\bf Suggerimento}: per realizzare la \verb|bpcontinue()| si deve temporaneamente rimuovere il breakpoint, quindi
reinserirlo non appena lo slave ha eseguito una istruzione. Per intercettare questo evento si pu\`o abilitare
temporaneamente il single-step trap nel processo slave.
