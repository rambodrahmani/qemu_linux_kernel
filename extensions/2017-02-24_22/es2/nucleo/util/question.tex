Colleghiamo al sistema delle periferiche PCI di tipo \verb|ce|, con vendorID \verb|0xedce| e deviceID \verb|0x1234|.
Ogni periferica \verb|ce| usa 32 byte nello spazio di I/O a partire dall'indirizzo base specificato nel
registro di configurazione BAR0, sia $b$.

La periferiche \verb|ce| sono periferiche di ingresso in grado di operare in PCI Bus Mastering.
Sono inoltre in grado di eseguire tutte le necessarie trasformazioni da indirizzi virtuali a fisici,
utilizzando le stesse strutture dati della MMU del processore, purch\`e non incontrino bit P a
zero durante la traduzione.

I registri accessibili al programmatore, tutti di 4 byte, sono i seguenti:
\begin{enumerate}
  \item {\bf VPTRHI} (indirizzo $b$): parte pi\`u significativa dell'indirizzo virtuale di destinazione
  	(sempre 0 nei sistemi a 32bit);
  \item {\bf VPTRLO} (indirizzo $b+4$): parte meno significativa dell'indirizzo virtuale di destinazine;
  \item {\bf CNT}    (indirizzo $b+8$): numero di byte da trasferire;
  \item {\bf CR3}    (indirizzo $b+12$): indirizzo fisico del direttorio (32bit) o tabella di livello 4 (64bit);
  \item {\bf STS}    (indirizzo $b+16$): registro di stato;
  \item {\bf CMD}    (indirizzo $b+20$): registro di comando.
\end{enumerate}
Ogni volta che si scrive il valore 1 nel registro CMD, la periferica tenta di scrivere CNT byte in memoria
a partire dall'indirizzo virtuale contenuto in VPTRHI, VPTRLO. Gli indirizzi verranno tradotti utilizzando
la tabella di corrispondenza puntata dal registro CR3 dell'interfaccia. Se l'interfaccia incontra degli errori
durante la traduzione (per es. un bit P a zero), interrompe il traferimento e setta il bit 2 di STS.
In ogni caso la periferica invia una richiesta di interruzione al completamento dell'operazione
(o perch\'e non ha pi\`u byte da trasferire, o perch\'e ha riscontrato un errore).

Le interruzioni sono sempre abilitate. La lettura del registro di stato funziona da risposta alle richieste di interruzione.

Modificare i file \verb|io.s| e \verb|io.cpp| in modo da realizzare la primitiva
\begin{verbatim}
   bool cedmaread(natl id, natl quanti, char *buf)
\end{verbatim}
che permette di leggere \verb|quanti| byte dalla periferica numero \verb|id| (tra quelle di questo tipo), copiandoli
nel buffer \verb|buf|.  La primitiva restituisce
\verb|false| se il trasferimento \`e stato interrotto per errori, \verb|true| altrimenti.

Controllare tutti i problemi di Cavallo di Troia.

Per descrivere le periferiche \verb|ce| aggiungiamo le seguenti strutture dati al modulo I/O:
\begin{verbatim}
    struct des_ce {
        natw iVPTRHI, iVPTRLO, iCNT, iCR3, iSTS, iCMd;
        natl sync;
        natl mutex;
        bool error;
    }; 
    des_ce array_ce[MAX_CE];
    natl next_ce;
\end{verbatim}
La struttura \verb|des_ce| descrive una periferica di tipo \verb|ce| e contiene al suo interno gli indirizzi
dei vari registri, l'indice di un semaforo inizializzato a zero (\verb|sync|), l'indice di un semaforo
inizializzato a 1 (\verb|mutex|) e un booleano per memorizzare il successo o fallimento dell'ultima operazione.

I primi \verb|next_ce| elementi del vettore \verb|array_ce| contengono i descrittori, opportunamente inizializzati,
delle periferiche di tipo \verb|ce| effettivamente rilevate in fase di avvio del sistema. Ogni periferica \`e identificata
dall'indice del suo descrittore.

{\bf Nota}: il modulo sistema mette a disposizione del modulo di I/O le primitive 
\begin{verbatim}
   natl resident(void addr, natl quanti);
   void nonresident(natl id);
\end{verbatim}
La primitiva \verb|resident()| permette di caricare in memoria e rendere temporaneamente non rimpiazzabili tutte le pagine
virtuali (e le relative tabelle) che contengono gli indirizzi da \verb|addr| a \verb|addr|+\verb|quanti| (escluso).
La primitiva restituisce un identificatore di operazione che pu\`o essere successivamente passato a \verb|nonresident()|
per rendere nuovamente rimpiazzabili le pagine (e le tabelle) in questione. In caso di errore restituisce \verb|0xffffffff|.
