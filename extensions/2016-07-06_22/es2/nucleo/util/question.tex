Colleghiamo al sistema delle periferiche PCI di tipo \verb|ce|, con vendorID \verb|0xedce| e deviceID \verb|0x1234|.
Ogni periferica \verb|ce| usa 16 byte nello spazio di I/O a partire dall'indirizzo base specificato nel
registro di configurazione BAR0, sia $b$.

La periferiche \verb|ce| sono periferiche di ingresso in grado di generare interruzioni.
I registri accessibili al programmatore sono i seguenti:
\begin{enumerate}
  \item {\bf CTL} (indirizzo $b$, 1 byte): registro di controllo; il bit numero 0 permette di abilitare (1) o disabilitare (0)
    le richieste di interruzione;
  \item {\bf STS} (indirizzo $b+4$, 1 byte): registro di stato; il bit numero 0 vale 1 se e solo se il registro RBR contiene
    un dato non ancora letto;
  \item {\bf RBR} (indirizzo $b+8$, 1 byte): registro di lettura;
\end{enumerate}
L'interfaccia genera una interruzione se le interruzioni sono abilitate e il registro RBR contiene un valore non ancora letto.
L'interfaccia non presenta nuovi valori in RBR se questo ne contiene uno non ancora letto, quindi la lettura di RBR funge
da risposta alla richiesta di interruzione.

Vogliamo fornire all'utente una primitiva 
\begin{verbatim}
   ceread(natl id, char *buf, natl& quanti, char stop)
\end{verbatim}
Il parametro \verb|id| identifica una delle periferiche \verb|ce| installate.
La primitiva permette di leggere da tale periferica una sequenza di byte che termina con il carattere \verb|stop| passato come quarto
argomento. I byte letti saranno scritti a partire dall'indirizzo \verb|buf|. Il parametro \verb|quanti| \`e usato
sia come argomento di ingresso che di uscita: in ingresso l'utente specifica il numero massimo di byte da leggere 
(anche se \verb|stop| non \`e stato ricevuto) e in uscita la primitiva dice all'utente il numero di byte che sono stati
effettivamente letti (che pu\`o essere inferiore al massimo, quando si riceve \verb|stop|).

Per descrivere le periferiche \verb|ce| aggiungiamo le seguenti strutture dati al modulo I/O:
\begin{verbatim}
des_ce array_ce[MAX_CE];
natl next_ce;
\end{verbatim}
I primi \verb|next_ce| elementi del vettore \verb|array_ce| contengono i destrittori, opportunamente inizializzati,
delle periferiche di tipo \verb|ce| effettivamente rilevate in fase di avvio del sistema. Ogni periferica \`e identificata
dall'indice del suo descrittore. La struttura \verb|des_ce| deve essere definita dal candidato.

Modificare i file \verb|io.s| e \verb|io.cpp| in modo da realizzare la primitiva come descritto.
